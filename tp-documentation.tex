\documentclass{article}
\usepackage[a4paper,total={18cm,25.5cm}]{geometry}


\usepackage[T1]{fontenc}
\usepackage[utf8]{inputenc}

\usepackage{microtype,hyperref,array,enumitem,tikz}

\usepackage[main=ngerman,english]{babel}

\usepackage[hyperref,enumitem]{tikz-palettes}
\UsePalette{Crimson}
\UseTikzPalette{CleanPlate}

\usepackage{imakeidx}
\makeindex[title=Befehlsübersicht,columns=2,columnsep=0.75cm]\renewcommand{\indexname}{Befehlsübersicht}


\def\say#1{\glqq{#1}\grqq{}}

\def\cmd#1{\texttt{\paletteA{\textbackslash#1}}}
\def\cmdlink#1{\phantomsection\label{cmd:#1}\cmd{#1}}
\def\cmdref#1{\hyperref[cmd:#1]{\cmd{#1}}}

\def\arg#1{\textit{\paletteB{#1}}}
\def\manArg#1{\texttt{\{\,\arg{#1}\,\}}}
\def\optArg#1{\texttt{[\,\arg{#1}\,]}}

\def\secref#1{\hyperref[#1]{\ref*{#1} \nameref*{#1}}}

\long\def\imp#1{\emph{\paletteD{#1}}}

\newenvironment{command}[3][v1.0]{\leavevmode\\[-\baselineskip]\hspace*{-1.75em}\index{#2@\cmdref{#2}}\cmdlink{#2}$\,$#3\hfill{}\texttt{#1}\nopagebreak\smallskip\\*}{\bigskip\par{}}


\title{tikz-palettes.sty}
\author{Florian Sihler}
\date{\today}

\begin{document}

\maketitle

\begin{abstract}
    Dieses Paket dient der Handhabung von Stilpaletten für Ti\textit{k}Z. Es verwendet das ebenfalls mitgelieferte \paletteA{color-palettes}-Paket zur Realisierung der Definitionen. \imp{Das Paket \texttt{tikz}/\texttt{pgf}\footnote{\url{https://www.ctan.org/pkg/pgf}} muss zuvor geladen werden.}
\end{abstract}
% TODO: Zwilling: tikz-palettes

\tableofcontents

\section{Nutzermakros}

\begin{command}{DefineTikzPalette}{\manArg{Name}\manArg{A}\manArg{B}\manArg{C}\manArg{D}\manArg{writ}}
    Definiert die Palette \arg{Name} ungeachtet ob sie bereits existiert (sollte sie bereits existieren, wird sie überschrieben). Die folgenden vier Argumente \arg{A} bis \arg{D} müssen im folgenden Format präsentiert werden und definieren die jeweilige Farbe:
\begin{verbatim}
<name>: <style>
\end{verbatim}
Das Argument \arg{writ}, welches im selben Schema präsentiert werden muss, kann für das Setzen von Texten definiert werden. Hier als Beispiel die Definition von \emph{CleanPlate}:
\begin{verbatim}
\DefineTikzPalette{CleanPlate}
    {\paletteAname: paletteA,fill=shadeA}
    {\paletteBname: paletteB,fill=shadeB}
    {\paletteCname: paletteC,fill=shadeC}
    {\paletteDname: paletteD,fill=shadeD}
    {Schwarz: text=black}
\end{verbatim}
\imp{Der Name einer Palette darf nicht leer sein!}, die Paletten sind unabhängig von denen, welche in \paletteA{color-palettes} definiert werden.
Das Paket definiert passende Paletten, welche in \secref{sec:predefined-palletes} präsentiert werden.

Die Stile stehen anschließend unter den Bezeichnern \texttt{tikzA}, \texttt{tikzB}, \texttt{tikzC} und \texttt{tikzD} zur Verfügung.
\end{command}

\begin{command}{CurrentTikzPalette}{}
    Dieser Befehl enthält den Bezeichner der aktuellen Palette. Er sollte nur \say{gelesen} und nicht geschrieben werden. Da in den meisten Dokumenten nur eine Palette eingesetzt wird, sollte seine Bedeutung (außerhalb des Pakets) eher gering sein.
\end{command}

\begin{command}{UseTikzPalette}{\manArg{Name}}
    Setzt die aktuelle Palette auf \arg{Name}, wirft einen Fehler wenn die Palette nicht existiert. Ein leerer Bezeichner sorgt für eine Löschung der aktuellen Paletten-Befehle. Die Tikz-Paletten unterscheiden sich von denen aus \paletteA{color-palettes}.
\end{command}

\begin{command}{ShowcaseTikzPalette}{\manArg{Name}}
    Ändert für den Befehl die Palette mittels \cmdref{UseTikzPalette} auf \arg{Name} und wechselt dann zur vorherigen Palette zurück. Zur Anzeige wird \cmdref{ShowcaseCurrentTikzPalette} verwendet.
\end{command}

\begin{command}{ShowcaseCurrentTikzPalette}{}
    Zeigt die aktuelle Palette in Form einer Tabelle an (die Gestalt kann sich über die Versionen durchaus ändern). Die Anzeige wird auch in \secref{sec:predefined-palletes} verwendet.
\end{command}

\begin{command}{tikz*name}{}
    Das Sternchen steht hierbei für die Bezeichner \emph{A}, \emph{B}, \emph{C} beziehungsweise \emph{D}. Der Befehl expandiert zum Namen der jeweiligen Farbe. So zum Beispiel (verwendet wird \emph{\CurrentTikzPalette}) wird \cmd{tikzAname} zu \say{\tikzAname}.
\end{command}

\begin{command}{writname}{}
    Analog zu \cmdref{tikz*name}, allerdings für die Schriftdefinition. Mit der Palette \emph{\CurrentTikzPalette} ergibt sich: \writname.
\end{command}

\begin{command}{AllTikzPalettes}{}
    Liefert eine Liste alle definierten Paletten: \say{\AllTikzPalettes}.
\end{command}


\section{Paketoptionen}
\label{sec:packetoptions}Das Paket selbst bietet keine besonderen Optionen, kann aber die Optionen für \paletteA{color-palettes} akzeptieren, die dann entsprechend weitergereicht werden.
\makeatletter
\section{Versionierung}
\subsection{Version 1.0}
Grundlegende Kommandos hinzugefügt. Darunter: \cmdref{DefineTikzPalette}, \cmdref{CurrentTikzPalette}, \cmdref{UseTikzPalette}, \cmdref{ShowcaseTikzPalette} und \cmdref{ShowcaseCurrentTikzPalette}.
\clearpage
\section{Vordefinierte Paletten}
\label{sec:predefined-palletes}\hbox{}\vfill
\bgroup
\footnotesize\begin{center}
    \begin{tabular}{cc}
        \ShowcaseTikzPalette{CleanPlate} & \ShowcaseTikzPalette{DeepInk} \\[8em] \ShowcaseTikzPalette{CleanStripes} & \ShowcaseTikzPalette{Blueprint} \\[8em]
        \ShowcaseTikzPalette{MajorMark} & \ShowcaseTikzPalette{BlueprintStripes} \\[8em]
    \end{tabular}
\end{center}
\egroup
\vfill\hbox{}
\clearpage\appendix
\addcontentsline{toc}{section}{\indexname}
\printindex
\end{document}