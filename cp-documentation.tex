\documentclass{article}
\usepackage[a4paper,total={18cm,25.5cm}]{geometry}


\usepackage[T1]{fontenc}
\usepackage[utf8]{inputenc}

\usepackage{microtype,hyperref,array,enumitem}

\usepackage[main=ngerman,english]{babel}

\usepackage[hyperref,enumitem]{color-palettes}

\UsePalette{IceLake}

\def\say#1{\glqq{#1}\grqq{}}

\def\cmd#1{\texttt{\paletteA{\textbackslash#1}}}
\def\cmdlink#1{\phantomsection\label{cmd:#1}\cmd{#1}}
\def\cmdref#1{\hyperref[cmd:#1]{\cmd{#1}}}

\def\arg#1{\textit{\paletteB{#1}}}
\def\manArg#1{\texttt{\{\,\arg{#1}\,\}}}
\def\optArg#1{\texttt{[\,\arg{#1}\,]}}

\def\secref#1{\hyperref[#1]{\ref*{#1} \nameref*{#1}}}

\long\def\imp#1{\emph{\paletteD{#1}}}

\newenvironment{command}[3][v1.0]{\leavevmode\\[-\baselineskip]\hspace*{-1.75em}\cmdlink{#2}$\,$#3\hfill{}\texttt{#1}\nopagebreak\smallskip\\*}{\bigskip\par{}}


\title{color-palettes.sty}
\author{Florian Sihler}
\date{\today}

\begin{document}

\maketitle

\begin{abstract}
    Dieses Paket dient der Handhabung von Farbpaletten. Es liefert Makros zur einfachen Definition vierfarbiger Paletten, die sich an die jeweilige Umgebung anpassen und auch
    gewechselt werden können. 
\end{abstract}
% TODO: Zwilling: tikz-palettes

\tableofcontents

\section{Nutzermakros}

\begin{command}{DefinePalette}{\manArg{Name}\manArg{A}\manArg{B}\manArg{C}\manArg{D}}
    Definiert die Palette \arg{Name} ungeachtet ob sie bereits existiert (sollte sie bereits existieren, wird sie überschrieben). Die folgenden 4 Argumente \arg{A} bis \arg{D} müssen im folgenden Format präsentiert werden und definieren die jeweilige Farbe:
\begin{verbatim}
<noun>,<adjective>: <format>(<specification>)
\end{verbatim}
Exemplarisch sei hier die Definition von \emph{IceLake} präsentiert:
\begin{verbatim}
\DefinePalette{IceLake}
    {Dunkelblau,dunkelbläulich: RGB(049, 068, 095)}
    {Blau,bläulich:             RGB(082, 115, 143)}
    {Hellblau,hellbläulich:     RGB(113, 151, 172)}
    {Rot,rötlich:               RGB(137, 064, 075)}
\end{verbatim}
\imp{Der Name einer Palette darf nicht leer sein!}

Die Farben stehen anschließend unter den Bezeichnern \texttt{paletteA}, \texttt{paletteB}, \texttt{paletteC} und \texttt{paletteD} zur Verfügung. Weiter wird für jede Farbe \texttt{shadeA}, \texttt{shadeB}, \texttt{shadeC} und \texttt{shadeD} als Farbe erzeugt. Es sind die helleren Varianten in \secref{sec:predefined-palletes}.
\end{command}

\begin{command}{CurrentPalette}{}
    Dieser Befehl enthält den Bezeichner der aktuellen Palette. Er sollte nur \say{gelesen} und nicht geschrieben werden. Da in den meisten Dokumenten nur eine Palette eingesetzt wird, sollte seine Bedeutung (außerhalb des Pakets) eher gering sein.
\end{command}

\begin{command}{UsePalette}{\manArg{Name}}
    Setzt die aktuelle Palette auf \arg{Name}, wirft einen Fehler wenn die Palette nicht existiert. Ein leerer Bezeichner sorgt für eine Löschung der aktuellen Paletten-Befehle.
\end{command}

\begin{command}{ShowcasePalette}{\manArg{Name}}
    Ändert für den Befehl die Palette mittels \cmdref{UsePalette} auf \arg{Name} und wechselt dann zur vorherigen Palette zurück. Zur Anzeige wird \cmdref{ShowcaseCurrentPalette} verwendet.
\end{command}

\begin{command}{ShowcaseCurrentPalette}{}
    Zeigt die aktuelle Palette in Form einer Tabelle an (die Gestalt kann sich über die Versionen durchaus ändern). Die Anzeige wird auch in \secref{sec:predefined-palletes} verwendet.
\end{command}

\begin{command}{palette*name}{}
    Das Sternchen steht hierbei für die Bezeichner \emph{A}, \emph{B}, \emph{C} beziehungsweise \emph{D}. Der Befehl expandiert zum Namen der jeweiligen Farbe. So zum Beispiel (verwendet wird \CurrentPalette) wird \cmd{paletteAname} zu \say{\paletteAname}.
\end{command}

\begin{command}{palette*adj}{}
    Das Sternchen steht hierbei für die Bezeichner \emph{A}, \emph{B}, \emph{C} beziehungsweise \emph{D}. Der Befehl expandiert zum adjektiv der jeweiligen Farbe. So zum Beispiel (verwendet wird \CurrentPalette) wird \cmd{paletteAadj} zu \say{\paletteAadj}.
\end{command}

\begin{command}[v1.1]{palette*}{\manArg{text}}
    Das Sternchen steht hierbei für die Bezeichner \emph{A}, \emph{B}, \emph{C} beziehungsweise \emph{D}. Der Befehl ist eine Kurzform für \texttt{\cmdref{textcolor}\{palette*\}\manArg{Text}} So ergibt \texttt{\cmd{paletteA}\{Hallo Welt\}} folgendes Ergebnis: \paletteA{Hallo Welt}.
\end{command}


\section{Paketoptionen}
\label{sec:packetoptions}Das Paket kann (\texttt{v1.1}) mit den Optionen \texttt{hyperref}/\texttt{nohyperref} geladen werden. Die
Angabe von \texttt{hyperref} fordert, dass das gleichnamige Paket geladen ist und
setzt die Linkfarben entsprechend der Palette neu. Beispiel:
{\UsePalette{SourCandy}\hyperref[sec:packetoptions]{Link in \CurrentPalette}}, \hyperref[sec:packetoptions]{Link in \CurrentPalette} und {\UsePalette{BloodyGrass}\hyperref[sec:packetoptions]{Link in \CurrentPalette}}.

Ebenso gibt es die Paketoptionen (\texttt{v1.1}) \texttt{enumitem}/\texttt{noenumitem} die bei geladenem gleichnamigen Paket die Gestalt\footnote{Nach aktuellem Stand wird automatisch \texttt{amssymb} geladen.} von \emph{itemize} verändern:
\begin{center}
    \begin{tabular}{*3{m{0.3\linewidth}}}
        \begin{itemize}
            \item Itemize in
            \item Palette \CurrentPalette
            \item \ldots
        \end{itemize} & {\UsePalette{SourCandy}\begin{itemize}
            \item Itemize in
            \item Palette \CurrentPalette
            \item \ldots
        \end{itemize}} & {\UsePalette{BloodyGrass}\begin{itemize}
            \item Itemize in
            \item Palette \CurrentPalette
            \item \ldots
        \end{itemize}}
    \end{tabular}
\end{center}

\section{Versionierung}
\subsection{Version 1.0}
Grundlegende Kommandos hinzugefügt. Darunter: \cmdref{DefinePalette}, \cmdref{CurrentPalette}, \cmdref{UsePalette}, \cmdref{ShowcasePalette} und \cmdref{ShowcaseCurrentPalette}.
\subsection{Version 1.1}
Die \hyperref[sec:packetoptions]{Paketoptionen} \texttt{hyperref} und \texttt{enumitem} wurden hinzugefügt.

\clearpage
\section{Vordefinierte Paletten}
\label{sec:predefined-palletes}\hbox{}\vfill
\bgroup
\footnotesize\begin{center}
    \begin{tabular}{cc}
        \ShowcasePalette{Water} & \ShowcasePalette{GraySun} \\[6em]
        \ShowcasePalette{ClayGrass} & \ShowcasePalette{PurpleCoin} \\[6em]
        \ShowcasePalette{Moonshine} & \ShowcasePalette{BloodyGrass} \\[6em]
        \ShowcasePalette{SourCandy} & \ShowcasePalette{PurpleSand} \\[6em]
        \ShowcasePalette{IceLake} & \ShowcasePalette{Crimson} \\[6em]
        \ShowcasePalette{GrayPrint} & \ShowcasePalette{Rainbow} \\[6em]
        \ShowcasePalette{Legacy} & \ShowcasePalette{LegacyPrint}
    \end{tabular}
\end{center}
\egroup
\vfill\hbox{}

\end{document}